%!TEX TS-program = xelatex 
%!TEX TS-options = -synctex=1 -output-driver="xdvipdfmx -q -E"
%!TEX encoding = UTF-8 Unicode
%
%  sound_and_image
%
%  Created by Mark Eli Kalderon on 2017-03-29.
%  Copyright (c) 2017. All rights reserved.
%

\documentclass[12pt]{article} 

% Definitions
\newcommand\mykeywords{audition, sound, image}
\newcommand\myauthor{Mark Eli Kalderon}
\newcommand\mytitle{Sound and Image}

% Packages
\usepackage{geometry} \geometry{a4paper} 
% \usepackage{txfonts}
\usepackage{enumerate}
\usepackage{setspace}
% \doublespace % Uncomment for doublespacing if necessary
% \usepackage{epigraph} % optional

% XeTeX
\usepackage[cm-default]{fontspec}
\usepackage{xltxtra,xunicode}
\defaultfontfeatures{Scale=MatchLowercase,Mapping=tex-text}
\setmainfont{Hoefler Text}

% Bibliography
\usepackage[round]{natbib}

% Title Information
\title{\mytitle}
\author{\myauthor}
\date{} % Leave blank for no date, comment out for most recent date

% PDF Stuff
\usepackage[plainpages=false, pdfpagelabels, pdftitle={\mytitle}, pdfauthor={\myauthor}, pdfkeywords={\mykeywords}, xetex, unicode=true]{hyperref} 

%%% BEGIN DOCUMENT
\begin{document}

% Title Page
\maketitle
% \vskip 2em \hrule height 0.4pt \vskip 2em
% \epigraph{} % optional; make sure to uncomment \usepackage{epigraph}
\begin{abstract} % optional
\noindent We hear sounds, and their sources, and their audible qualities. Sounds and their sources are essentially dynamic entities, not wholly present at any given moment, but unfolding through their temporal interval. Sounds and their sources, essentially dynamic entities, are the bearers or \emph{susbtrata} of audible qualities. Audible qualities are qualities essentially sustained by activity. The only bearers of audible qualities present in auditory experience are essentially dynamic entities. Bodies are not, in this sense, essentially dynamic entities and so are not present in our auditory experience. Though absent in auditory experience, we may, nonetheless, attend to bodies in audition, when an audible sound-generating event in which they participate presents a dynamic aural image of them.

\end{abstract}

% Layout Settings
\setlength{\parindent}{1em}

% Main Content

Suppose that the bearers of audible qualities, such as pitch and timbre, are essentially dynamic entities such as events or processes, not wholly present at any moment, but unfolding through time. If we suppose further that only audible qualities and their bearers are present in auditory experience then a puzzle arises. For we seem to be able to attend to bodies in audition. We can listen to an animal’s approach, say. But bodies are not essentially dynamic entities, nor do they inhere in such. But only audible qualities and their bearers are present in auditory experience. Which means that bodies, being neither, are absent. But if bodies are absent in auditory experience how may we attend to them in audition?

I believe that this puzzle may be resolved without rejecting the central claim upon which it rests, that the bearers of audible qualities are essentially dynamic entities. In the first part of this talk, I try to motivate that claim. I do so both directly and indirectly by criticizing an alternative. I end with a speculative resolution of our puzzle.

What are the bearers of audible qualities? What are the kinds of things in which such qualities inhere? I claim that the bearers of audible qualities are essentially dynamic entities such as events, as opposed to bodies. If the bearers of audible qualities are essentially dynamic entities, then audible qualities are qualities essentially sustained by activity.

Consider the following analogy: Colors are spatially extended, at least in the sense of being instanced only by spatially extended things. We can imagine smaller and smaller things being colored, but we cannot conceive of a thing without extension exhibiting color. Audible qualities are temporally extended, at least in the sense of being instanced only by temporally extended things. We can imagine hearing briefer and briefer occurrences of pitch, but we cannot conceive of a thing without duration exhibiting pitch. The temporal dimension of the bearers of audible qualities is not exhausted by their having a beginning and end, sounds have a distinctive way of being in time. Like events, at least as the three-dimensionalist conceives of them, sounds unfold in time. Unlike states that are wholly present whenever they obtain, sounds are not wholly present at every moment of their sounding. They are spread over the interval of time through which they unfold.

% Perhaps some sounds, such as the sound of the wind, or the roar of a waterfall, are more like processes than events. However, that distinction is not presently relevant, and at any rate, processes, like events, are essentially dynamic entities, though perhaps in their own way.

% That sounds are not wholly present at any moment of their sounding precludes them from being wholly present in auditory experience at any moment of their hearing. If we further assume that perceptual experience only presents what could be present at any given moment, then a puzzle about the very possibility of audition arises, as Prichard observes:
% \begin{quote}
% 	We should ordinarily be said to hear certain noises, e.g. the sound a bell or the note of a bird. But any sound has duration, however short. If so, how can it ever be true that we apprehend by way of hearing—or more generally perceiving—can only exist at the moment of hearing, and ex hypothesi, at least part of the sound said to be heard is over at the moment of hearing, and strictly speaking it is a  over. And the difficulty seems a double one. For since a sound has duration, it cannot exist at the moment of hearing, and therefore we cannot hear a present sound—for there is no such thing. And if it is over and so not existing at the moment when we are said to hear it, it cannot be heard. Therefore, it seems, it is impossible hear a sound. (Prichard, 1950b, 47)
% \end{quote}
% The most straightforward way to deal with this puzzle is to abandon the principle that generates it---that perceptual experience only presents what could be present at any given moment. If we abandon this principle, then we may conclude that since sounds are spread over time, their sensory presentation must also be. Auditory experience unfolds with its object. We listen along with what we hear. So auditory presentation, due to the distinctive temporal nature of sound, has duration. Auditory presentation is the disclosure of a sound unfolding through its temporal interval. It discloses its object, then, over time, just like haptic presentation. However, whereas haptic perception discloses relatively static features such as texture and temperature, sounds, by contrast, are essentially dynamic entities, not wholly present at any moment of their existence but unfolding in time.

% Sounds may be particular events, and so have a mode of being that suffices to distinguish them from entities belonging to other ontological categories such as bodies and states, but what of other audibilia? Must all audible objects unfold through time? Or is this just a feature of, in Peripatetic vocabulary, the proper objects of audition?

We hear sounds, and their sources, but we also hear their audible qualities. Audible qualities, such as pitch, are not essentially dynamic entities unfolding through time. Rather, their mode of being is more akin to the mode of being of states. Nevertheless, sounds and their sources, conceived as essentially dynamic entities, not wholly present at any moment, but unfolding through time, are bearers of audible qualities. That the bearer of an audible quality is an essentially dynamic entity is manifest in the conditions under which that quality may be instantiated. There is no instantaneous pitch since there is nothing instantaneous to instantiate it. For pitch to exist, it must persist over time. And that is because audible qualities are qualities essentially sustained by activity. The audible qualities of a sound will vary and extinguish as the sound’s activity varies and extinguishes. Sounds without audible qualities would be inaudible, but audible qualities without sound would simply not be (or at least, those audible qualities that modify sounds, as opposed to other audible \emph{substrata}, such as sources). Audible qualities, while not essentially dynamic entities, are qualities that audible activity gives rise to. They are qualities of audible events or processes or phases of these.

% If audibilia means all that we can hear, and we can hear a sound’s pitch, then not all audibilia are essentially dynamic entities. There may be no instantaneous pitch, pitch may be a quality essentially sustained by activity, but pitch, as a quality, does not unfold through time. Of all that we hear, some of what we hear is existentially and ontologically prior to other things that we hear. The audible qualities of a sound will vary and extinguish as the sound’s activity varies and extinguishes. Sounds are existentially and ontologically prior to their audible qualities because sounds are the bearers or substrata of audible qualities. At best, then, the claim should be that the existentially and ontologically prior substrata of audible qualities are essentially dynamic entities, not wholly present at any moment, but unfolding through time. So conceived, audible qualities would be qualities essentially sustained by activity



\citet{Kulvicki:2008aa} has argued that the bearers of audible qualities are not events but bodies. Like Aristotle, \emph{De anima} 2 11 422\( ^{b} \)31–32, he accepts that an audible quality has a bearer. Being quality instances, they must inhere in something upon which they existentially and ontologically depend. However, unlike Aristotle, he denies that sound is the bearer of audible qualities. Audible qualities inhere in bodies but these bodies are not themselves sounds. Rather, they are ordinary material substances. Instead, sounds are the audible qualities that inhere in these bodies and are manifest in their audible activity. Thus, like \citet{Pasnau:1999ss} and \citet{Leddington:2014aa}, Kulivicki endorses a broadly Lockean metaphysics that identifies sounds with audible qualities. However, it is not the Lockean metaphysics of sound that is our present focus, but whether bodies are bearers of audible qualities.

Bodies have resonant modes determined by their material structure. Because of their resonant modes, bodies are disposed to vibrate at certain natural frequencies when “thwacked”. According to Kulvicki, the sound a body has, an audible quality of it, is the stable disposition to vibrate when thwacked. Just as the energy of the illuminant reveals the colors of things to sight, the energy of thwacking reveals the sounds of things to hearing. And just as bodies retain their colors even when unilluminated, bodies retain their sound even when unthwacked. The stable disposition to vibrate when thwacked is a sound that a body has. Not every sound that a thing makes is a sound that a thing has. Stereo speakers when thwacked produce a dull thud, but when played they can make a wide variety of sounds.

Why think that sounds are qualities of material bodies that are associated with their natural frequencies? Kulvicki provides an argument from perceptual constancy that, while not conclusive, is meant to speak strongly in favor of his view. Kulvicki draws our attention to an interesting feature of speech perception, out ability to recognize voices. A speaker’s voice will vary in pitch, timbre, and so on, as they speak. And yet despite these variable auditory appearances, we seem to be presented with a constant voice in our experience of their speech. This is due, in part, to the resonant modes of the special parts of the speaker involved in speech production, such as their vocal cords and nasal cavities. And this is just the kind of auditory constancy one would expect if the sounds that we hear were stable dispositions of objects to vibrate when thwacked.

We have our voices. At least as we ordinarily speak. But do we have them, as well, in Kulvicki’s extraordinary sense? Or are they sounds that we make but do not have? The sound of a stereo speaker playing is a sound that it makes but does not have. I suspect that a person’s voice is more like the sound of a stereo speaker playing than the sound that it makes when thwacked. Through a series of unfortunate events, I have first hand experience of what I sound like when thwacked. I can attest it sounds nothing like my voice. Like a stereo speaker, I produce a dull thud when thwacked. When playing, a stereo speaker produces the sounds that it makes but does not have by an internal activity driving the vibration of special parts of it. When speaking, I produce the sounds that I do by an internal activity driving the vibration of special parts of myself. Are these not sounds that I make but do not have? If the sound of my voice is something that I make but do not have, then its being the constant element in an auditory experience provides no reason for thinking that sounds are stable dispositions to vibrate when thwacked since these are sounds that bodies were meant to have rather than make. 

Kulvicki is right to emphasize that auditory experience can disclose the stable dispositions of bodies to vibrate at their natural frequencies and so auditorily manifest, albeit partially and imperfectly, material properties of those bodies. But in hearing that, is what we hear a sound or its source? Suppose that we hear sounds and their sources. And suppose that the sources that we hear are sound-generating events. A body’s participation, if not the body itself, is part of the audible structure of that event. And those aspects of the body relevant to its participation in the event are reflected, partially and imperfectly, in its audible structure. Stable dispositions of bodies to vibrate at their natural frequencies given their resonant modes as determined by their material structure are aspects of bodies relevant to their participation in audible activities, such as being thwacked. When Dr Johnson, outside of the church in Harwich, kicked the stone, his boot rebounding despite its mighty force, the stone was well and truly thwacked. Doubtless, it could be heard as well as felt. And Dr Johnston could hear, as well as feel, that it was a stone, and not a log, that he was kicking. He could hear his boot kicking a stone as opposed to a log because of their distinctive timbre. Their different resonant modes are relevant to their participation in audible activities such as being kicked. Kulvicki is right to emphasize that auditory experience can disclose the stable dispositions of bodies to vibrate at their natural frequencies and so auditorily manifest, albeit partially and imperfectly, material properties of these bodies. But he was wrong to suggest that this requires bodies to be the bearers of audible qualities. 

We hear sounds and their sources. These are essentially dynamic entities, not wholly present at any given moment, but unfolding through time, or so I claim. Sounds and their sources have audible qualities, qualities essentially sustained by activity. Are sounds and their sources, as well as their audible qualities, really all that we can hear?

According to \citet[4]{Broad:1952kx}, we ordinarily speak of hearing bodies. So when Big Ben strikes the time, and is in earshot, we may say that we can hear Big Ben. However, Broad concedes little in acknowledging this point of usage since he also observes that it takes but a little pressure to convince “the plainest of plain men” that “hearing Big Ben” is shorthand for hearing the striking of Big Ben. If we accept Broad’s suggestion, then we only hear Big Ben insofar as it is a participant in a sound-generating event or process. And when we do, what we strictly speaking hear is Big Ben’s striking and not Big Ben, that is, not the body, but an event the body participates in. We hear not the body in a condition of activity, but the activity of the body.

% Sounds are events or processes, and their sources that we hear are the events and processes that generate those sounds. Do sources have audible qualities? Berkeley denied that they did. Only the sounds that they produce have audible qualities. And if sources lack audible qualities, then they are inaudible. Or so Berkeley contends. If sources, pace Berkeley, have audible qualities, then they are existentially and ontologically prior to the audible qualities for which they are bearers or substrata. Sources of sound are plausibly the bearers of timbre. Moreover, sources, being sound-generating events or processes, are essentially dynamic entities. Such a view would be a step closer to vindicating the general claim that among audibilia, the bearers or substrata of audible qualities have the distinctive temporal mode of being of events or processes. Full vindication would require further assurance that sounds and their sources alone have this status.

Allow me to engage in speculation about a hypothetical sense in which we may be said to hear bodies consistent with the principle, if true, that audition only presents bearers of audible qualities with the distinctive temporal mode of being of events or processes.

Both sounds and the sources that we hear are like events in that they are not wholly present at every moment of their occurrence. Perhaps this is a general feature of the bearers of audible qualities present in audition. Perhaps for a bearer of an audible quality to be present in auditory experience it must have a particular temporal mode of being, it must unfold through time. This would preclude, by their very nature, entities such as bodies from being present in auditory experience. First, by hypothesis, bodies lack the requisite temporal mode of being of bearers of audible qualities. And second, bodies do not inhere in essentially dynamic entities the way that audible qualities do. But if what is present in auditory experience is either essentially dynamic or an audible quality that the essentially dynamic \emph{substratum} gives rise to, then bodies are not present in auditory experience. Earlier we noted Broad’s helpful suggestion that perhaps “hearing Big Ben” is elliptical for hearing Big Ben’s striking.

As plausible as this may be, a worry may still persist. One of the uses to which audition may be put is to track a body’s progress through the natural environment. We can listen to an animal’s approach, say. And it might be thought that we are attending to the animal in audition in listening to them. Moreover, it might seem insufficient for the body to be attended to that an event in which that body participates is present in auditory experience. Not every part of a visible body is seen, so why assume that every participant of an audible event is heard? How can we listen out for bodies even though they are precluded from being present in auditory experience?

Bodies may not be present in auditory experience, but perhaps they figure in auditory experience in another way, if not as the intentional object of experience, then something very much like it. Bodies are, on the speculative hypothesis that we are entertaining, not present in auditory experience. Thus bodies are absent in auditory experience. And yet we can attend to bodies in audition. How could this be?

Aristotle uses this kind of puzzle or aporia about presence in absence to argue for, as we might put it, the intentional character of memory (\emph{De memoria et reminiscentia} 450\( ^{a} \)25--451\( ^{a} \)1, for discussion see \citealt{Sorabji:2004qa}). The Peripatetic response to the puzzle is to straightforwardly accept the claim of absence and reinterpret what purported to be a presentation instead as a kind of re-presentation. When one remembers Corsicus in his absence one contemplates a \emph{phantasma} caused by a previous perception of Corsicus and one conceives of the \emph{phantasma} as a likeness and reminder of Corsicus as he was perceived. How might the Peripatetic response, so abstractly described, be applied to the perceptual case of attending to bodies in audition?

Perhaps what is present in auditory experience may constitute a natural image of what is absent. That is, perhaps we can understand hearing the body’s sound-generating activity as providing the listener with a dynamic aural image of the body otherwise absent in audition. It is an image, indeed, as I have suggested, a natural image, like a fossil or a footprint. But unlike paradigmatic images it is not a visual image but an aural image. And while visual images are static, aural images, if such there be, would be dynamic as befitting their aural character. Hearing Big Ben’s striking, while not the presentation of Big Ben in auditory experience, would nevertheless provide the listener with a dynamic aural image of Big Ben. We do not so much as hear Big Ben in a condition of activity as we hear Big Ben in its audible activity. In order for this to be so, the auditory presentation of a sound-generating event must involve at least the partial disclosure of the event’s participants. Audition partially discloses an event’s participant by presenting it's participation in the audible event. It is the body’s participation in the event, and not the body per se, that is part of the event’s audible structure. The disclosure of such audible structure is partial. Only those aspects of the body that are manifest in its participation in the audible event are disclosed, and perhaps only some of those. Furthermore, there is no guarantee that if a perceiver hears an event, they hear each of its participants, if any. But that is consistent with audition, in certain circumstances of perception, partially disclosing at least some of the participants in the unfolding audible event. It is only if we can hear Big Ben’s participation in its striking that we can use that hearing to attend to Big Ben. It is only if we can hear Big Ben’s participation, can that hearing provide us with a dynamic aural image of Big Ben and its activities that we exploit in attending to Big Ben in audition.

\bibliographystyle{plainnat}
\bibliography{Philosophy}

\end{document}